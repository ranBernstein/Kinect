\usepackage[margin=1.1in]{geometry}
\usepackage{graphicx}
\usepackage{csvsimple}
\usepackage{varwidth}
\usepackage{array}
\usepackage{float}
\usepackage{amsmath}
\usepackage{pgfplotstable}
\usepackage{amsmath }
\usepackage[T1]{fontenc}
\usepackage[compact]{titlesec}
\usepackage{authblk}

\author[1]{Bernstein Ran}
\author[2]{Shafir Tal}
\author[3]{Tsachor Rachelle}
\author[4]{Studd Karen}
\author[1]{Schuster Assaf}
\affil[1]{Department of Computer Science, Technion I.I.T, Haifa, Israel}
\affil[2]{The Graduate School of Creative Arts Therapies, University of Haifa}
\affil[3]{School of Theatre \& Music, The University of Illinois at Chicago}
\affil[4]{School of Dance, George Mason University}

\begin{document}
\title{Laban Movement Analysis using Kinect}
\date{}
\maketitle
\begin{quote}{``Man moves in order to satisfy a need.`` ---\textup{Rudolph Laban}}
\end{quote}
\begin{abstract}
\textbf{Laban Movement Analysis (LMA) is a method for describing, interpreting
and documenting all varieties of human movement. 
Analyzing movements using LMA is advantageous over kinematic description, 
as it captures their qualitative aspects in addition to the quantitative. 
Thus, in recent years, LMA is increasingly becoming the preferred method for movement analysis. 
In this study we developed a Machine Learning (ML) method for recognizing Laban qualities from 
a markerless Motion Capture (MOCAP) camera --- Microsoft's Kinect. 
We believe that we are the first succeeded identifying LMA with a ubiquitous
sensor.
There no papers similar enough to ours for a performance comparison,
but our work obtained a recall and precision rate of about 60\%  averaged over
the qualities, result that is a solid foundation for a future work, and even a
success by itself.}
\end{abstract}
\end{document}